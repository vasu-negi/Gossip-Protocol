\PassOptionsToPackage{unicode=true}{hyperref} % options for packages loaded elsewhere
\PassOptionsToPackage{hyphens}{url}
%
\documentclass[]{article}
\usepackage{lmodern}
\usepackage{amssymb,amsmath}
\usepackage{ifxetex,ifluatex}
\usepackage{fixltx2e} % provides \textsubscript
\ifnum 0\ifxetex 1\fi\ifluatex 1\fi=0 % if pdftex
  \usepackage[T1]{fontenc}
  \usepackage[utf8]{inputenc}
  \usepackage{textcomp} % provides euro and other symbols
\else % if luatex or xelatex
  \usepackage{unicode-math}
  \defaultfontfeatures{Ligatures=TeX,Scale=MatchLowercase}
\fi
% use upquote if available, for straight quotes in verbatim environments
\IfFileExists{upquote.sty}{\usepackage{upquote}}{}
% use microtype if available
\IfFileExists{microtype.sty}{%
\usepackage[]{microtype}
\UseMicrotypeSet[protrusion]{basicmath} % disable protrusion for tt fonts
}{}
\IfFileExists{parskip.sty}{%
\usepackage{parskip}
}{% else
\setlength{\parindent}{0pt}
\setlength{\parskip}{6pt plus 2pt minus 1pt}
}
\usepackage{hyperref}
\hypersetup{
            pdfborder={0 0 0},
            breaklinks=true}
\urlstyle{same}  % don't use monospace font for urls
\usepackage{graphicx,grffile}
\makeatletter
\def\maxwidth{\ifdim\Gin@nat@width>\linewidth\linewidth\else\Gin@nat@width\fi}
\def\maxheight{\ifdim\Gin@nat@height>\textheight\textheight\else\Gin@nat@height\fi}
\makeatother
% Scale images if necessary, so that they will not overflow the page
% margins by default, and it is still possible to overwrite the defaults
% using explicit options in \includegraphics[width, height, ...]{}
\setkeys{Gin}{width=\maxwidth,height=\maxheight,keepaspectratio}
\setlength{\emergencystretch}{3em}  % prevent overfull lines
\providecommand{\tightlist}{%
  \setlength{\itemsep}{0pt}\setlength{\parskip}{0pt}}
\setcounter{secnumdepth}{0}
% Redefines (sub)paragraphs to behave more like sections
\ifx\paragraph\undefined\else
\let\oldparagraph\paragraph
\renewcommand{\paragraph}[1]{\oldparagraph{#1}\mbox{}}
\fi
\ifx\subparagraph\undefined\else
\let\oldsubparagraph\subparagraph
\renewcommand{\subparagraph}[1]{\oldsubparagraph{#1}\mbox{}}
\fi

% set default figure placement to htbp
\makeatletter
\def\fps@figure{htbp}
\makeatother


\date{}

\begin{document}

Project Report

\hypertarget{table-of-contents}{%
\subsection{Table of Contents}\label{table-of-contents}}

\begin{itemize}
\tightlist
\item
  \protect\hyperlink{table-of-contents}{Table of Contents}
\item
  \protect\hyperlink{build-process}{Build Process}
\item
  \protect\hyperlink{what-is-working}{What is Working}

  \begin{itemize}
  \tightlist
  \item
    \protect\hyperlink{gossip}{Gossip}

    \begin{itemize}
    \tightlist
    \item
      \protect\hyperlink{normal-scale}{Normal Scale}
    \item
      \protect\hyperlink{log-scale}{Log Scale}
    \end{itemize}
  \item
    \protect\hyperlink{pushsum}{PushSum}

    \begin{itemize}
    \tightlist
    \item
      \protect\hyperlink{linear-scale}{Linear Scale}
    \item
      \protect\hyperlink{logarithmic-scale}{Logarithmic Scale}
    \end{itemize}
  \end{itemize}
\item
  \protect\hyperlink{what-is-the-largest-network-you-managed-to-deal-with-for-each-type-of-topology-and-algorithm}{What
  is the largest network you managed to deal with for each type of
  topology and algorithm}

  \begin{itemize}
  \tightlist
  \item
    \protect\hyperlink{gossip-10k}{Gossip 10k}

    \begin{itemize}
    \tightlist
    \item
      \protect\hyperlink{normal-scale-10k}{Normal Scale 10k}
    \item
      \protect\hyperlink{log-scale-10k}{Log Scale 10k}
    \end{itemize}
  \item
    \protect\hyperlink{pushsum-10k}{PushSum 10k}

    \begin{itemize}
    \tightlist
    \item
      \protect\hyperlink{linear-scale-10k}{Linear Scale 10k}
    \item
      \protect\hyperlink{logarithmic-scale-10k}{Logarithmic Scale 10k}
    \end{itemize}
  \end{itemize}
\item
  \protect\hyperlink{interesting-finds}{Interesting finds}
\end{itemize}

\hypertarget{build-process}{%
\subsection{Build Process}\label{build-process}}

\begin{itemize}
\tightlist
\item
  unzip the compressed file using \texttt{unzip\ filename.zip}
\item
  \texttt{dotnet\ fsi\ -\/-langversion:preview\ proj2.fsx\ nodeNum\ topology\ protocol}
  to run script where \texttt{nodeNum} is the number of nodes you want
  to run topology for. \texttt{topology} can have values in
  {[}\texttt{line}, \texttt{full}, \texttt{2D}, \texttt{Imp2D}{]}. The
  protocol can have values either \texttt{gossip} or \texttt{push-sum}.
\end{itemize}

\hypertarget{what-is-working}{%
\subsection{What is Working}\label{what-is-working}}

We able to run all \texttt{line}, \texttt{full}, \texttt{2D}, and
\texttt{Imp2D} in any combination with \texttt{gossip} or
\texttt{push-sum} protocol. The convergence in Gossip protocol is
achieved when all the nodes have converged. A node is set to be
converged when it listens to the message for the 10th time. After
convergence, the node stops transmitting the message to its neighbor.
Once the network is converged i.e.~all nodes are converged, the total
time for convergence is printed out.

\hypertarget{gossip}{%
\subsubsection{Gossip}\label{gossip}}

\hypertarget{normal-scale}{%
\paragraph{Normal Scale}\label{normal-scale}}

\begin{itemize}
\item
  Line topology is the slowest to converge. This is due to the fact that
  it has only access to 2 neighbors (left and right node).
\item
  Full topology is the fastest to converge in all possible scenarios.
  This is because it is connected to all the nodes and convergence is
  faster to achieve in this scenario
\item
  As expected, 2D and imperfect 2D would achieve the convergence time in
  between the line and full with imperfect 2D performing slightly better
  or equal to 2D performance.
\end{itemize}

\textbf{Note:} All nodes weren't converging, on multiple runs we would
notice that the convergence rate of topology would be 80-90\%. This is
because of the fact that the structure breaks while converging and some
nodes are unreachable. In order to tackle this problem, we keep a track
of all nodes that aren't converged, select one non converged node at
random and keep sending messages till the time topology doesn't achieve
the convergence.

\begin{figure}
\centering
\includegraphics{./docs/gossip.png}
\caption{Gossip}
\end{figure}

\hypertarget{log-scale}{%
\paragraph{Log Scale}\label{log-scale}}

\begin{figure}
\centering
\includegraphics{./docs/gossip_log.png}
\caption{Gossip Log Scale}
\end{figure}

\hypertarget{pushsum}{%
\subsubsection{PushSum}\label{pushsum}}

The PushSum network works by sending message \texttt{s} and \texttt{w}
as parameter to an actor. The intial value of \texttt{s} is equal to the
index of the actor and \texttt{w\ =\ 1}. The propagation stops when an
actor's \texttt{s/w} ratio does not change for 3 times in a row (i.e
stays within limit of 10\^{}-10)

\textbf{Note:} Unlike gossip algorithms, the push sum algorithm is able
to converge in all topologies. We believe this is due to the fact we try
to reduce the value s/w ratio till it does stop changing the ratio for
consecutive three times. This allows in more messages being transmitted
compared to the gossip algorithm allowing the network to converge. This
also causes an increase in convergence time.

\hypertarget{linear-scale}{%
\paragraph{Linear Scale}\label{linear-scale}}

\begin{figure}
\centering
\includegraphics{./docs/pushsum.png}
\caption{PushSum}
\end{figure}

\hypertarget{logarithmic-scale}{%
\paragraph{Logarithmic Scale}\label{logarithmic-scale}}

\begin{figure}
\centering
\includegraphics{./docs/pushsum_log.png}
\caption{PushSum}
\end{figure}

\hypertarget{what-is-the-largest-network-you-managed-to-deal-with-for-each-type-of-topology-and-algorithm}{%
\subsection{What is the largest network you managed to deal with for
each type of topology and
algorithm}\label{what-is-the-largest-network-you-managed-to-deal-with-for-each-type-of-topology-and-algorithm}}

The largest network that we have managed to solve is 10k nodes for each
topology and algorithm. We decided to limit it to 10k node so that the
network could be compared. Notice that other topologies (full, 2D, and
imp2D) could run on larger nodes within a reasonable time frame.

The following are the result of running the topologies in different
combinations.

\hypertarget{gossip-10k}{%
\subsubsection{Gossip 10k}\label{gossip-10k}}

\hypertarget{normal-scale-10k}{%
\paragraph{Normal Scale 10k}\label{normal-scale-10k}}

\begin{figure}
\centering
\includegraphics{./docs/gossip_10k.png}
\caption{Gossip}
\end{figure}

\hypertarget{log-scale-10k}{%
\paragraph{Log Scale 10k}\label{log-scale-10k}}

\begin{figure}
\centering
\includegraphics{./docs/gossip_log_10k.png}
\caption{Gossip}
\end{figure}

\hypertarget{pushsum-10k}{%
\subsubsection{PushSum 10k}\label{pushsum-10k}}

\hypertarget{linear-scale-10k}{%
\paragraph{Linear Scale 10k}\label{linear-scale-10k}}

\begin{figure}
\centering
\includegraphics{./docs/pushsum_10k.png}
\caption{Gossip}
\end{figure}

\hypertarget{logarithmic-scale-10k}{%
\paragraph{Logarithmic Scale 10k}\label{logarithmic-scale-10k}}

\begin{figure}
\centering
\includegraphics{./docs/pushsum_log_10k.png}
\caption{Gossip}
\end{figure}

\hypertarget{interesting-finds}{%
\subsection{Interesting finds}\label{interesting-finds}}

\begin{itemize}
\item
  For the Gossip algorithm, Full topology beats hands down everyone else
  since every node is connected to every other node, the probability of
  receiving messages from other nodes increasing drastically. The
  blindspot (nodes that never converge) in the case of full topology is
  almost negligible. Therefore making it the best topology to spread the
  rumor.
\item
  In the case of PushSum, if we reduce the gradient of s/w (i.e.
  \texttt{delta}) 10\^{}-10 to 10\^{}-5, the topology convergence time
  is also reduced to half.
\item
  Line algorithm will perform poorly since the number of the neighbor is
  the smallest and thereby also increasing the chances of failure.
\end{itemize}

\end{document}
