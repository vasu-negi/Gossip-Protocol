\PassOptionsToPackage{unicode=true}{hyperref} % options for packages loaded elsewhere
\PassOptionsToPackage{hyphens}{url}
%
\documentclass[]{article}
\usepackage{lmodern}
\usepackage{amssymb,amsmath}
\usepackage{ifxetex,ifluatex}
\usepackage{fixltx2e} % provides \textsubscript
\ifnum 0\ifxetex 1\fi\ifluatex 1\fi=0 % if pdftex
  \usepackage[T1]{fontenc}
  \usepackage[utf8]{inputenc}
  \usepackage{textcomp} % provides euro and other symbols
\else % if luatex or xelatex
  \usepackage{unicode-math}
  \defaultfontfeatures{Ligatures=TeX,Scale=MatchLowercase}
\fi
% use upquote if available, for straight quotes in verbatim environments
\IfFileExists{upquote.sty}{\usepackage{upquote}}{}
% use microtype if available
\IfFileExists{microtype.sty}{%
\usepackage[]{microtype}
\UseMicrotypeSet[protrusion]{basicmath} % disable protrusion for tt fonts
}{}
\IfFileExists{parskip.sty}{%
\usepackage{parskip}
}{% else
\setlength{\parindent}{0pt}
\setlength{\parskip}{6pt plus 2pt minus 1pt}
}
\usepackage{hyperref}
\hypersetup{
            pdfborder={0 0 0},
            breaklinks=true}
\urlstyle{same}  % don't use monospace font for urls
\setlength{\emergencystretch}{3em}  % prevent overfull lines
\providecommand{\tightlist}{%
  \setlength{\itemsep}{0pt}\setlength{\parskip}{0pt}}
\setcounter{secnumdepth}{0}
% Redefines (sub)paragraphs to behave more like sections
\ifx\paragraph\undefined\else
\let\oldparagraph\paragraph
\renewcommand{\paragraph}[1]{\oldparagraph{#1}\mbox{}}
\fi
\ifx\subparagraph\undefined\else
\let\oldsubparagraph\subparagraph
\renewcommand{\subparagraph}[1]{\oldsubparagraph{#1}\mbox{}}
\fi

% set default figure placement to htbp
\makeatletter
\def\fps@figure{htbp}
\makeatother


\date{}

\begin{document}

Project Report

\hypertarget{table-of-contents}{%
\subsection{Table of Contents}\label{table-of-contents}}

\begin{itemize}
\tightlist
\item
  \protect\hyperlink{table-of-contents}{Table of Contents}
\item
  \protect\hyperlink{build-process}{Build Process}
\item
  \protect\hyperlink{what-is-working}{What is Working}
\item
  \protect\hyperlink{what-is-the-largest-network-you-managed-to-deal-with-for-each-type-of-topology-and-algorithm}{What
  is the largest network you managed to deal with for each type of
  topology and algorithm}
\end{itemize}

\hypertarget{build-process}{%
\subsection{Build Process}\label{build-process}}

\begin{itemize}
\tightlist
\item
  unzip the compressed file using \texttt{unzip\ filename.zip}
\item
  \texttt{dotnet\ fsi\ -\/-langversion:preview\ proj2.fsx\ nodeNum\ topology\ protocol}
  to run script where \texttt{nodeNum} is the number of nodes you want
  to run topology for. \texttt{topology} can have values in
  {[}\texttt{line}, \texttt{full}, \texttt{2D}, \texttt{Imp2D}{]}.
  protocol can have values either \texttt{gossip} or \texttt{push-sum}.
\end{itemize}

\hypertarget{what-is-working}{%
\subsection{What is Working}\label{what-is-working}}

We able to run all \texttt{line}, \texttt{full}, \texttt{2D} and
\texttt{Imp2D} in any combination with \texttt{gossip} or
\texttt{push-sum} protocol

\hypertarget{what-is-the-largest-network-you-managed-to-deal-with-for-each-type-of-topology-and-algorithm}{%
\subsection{What is the largest network you managed to deal with for
each type of topology and
algorithm}\label{what-is-the-largest-network-you-managed-to-deal-with-for-each-type-of-topology-and-algorithm}}

The largest network that we have managed to solve is 10k nodes for each
topology and algorithm.

\end{document}
